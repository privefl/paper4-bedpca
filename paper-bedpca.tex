%% LyX 1.3 created this file.  For more info, see http://www.lyx.org/.
%% Do not edit unless you really know what you are doing.
\documentclass[english, 12pt]{article}
\usepackage{times}
%\usepackage{algorithm2e}
\usepackage{url}
\usepackage{bbm}
\usepackage[T1]{fontenc}
\usepackage[latin1]{inputenc}
\usepackage{geometry}
\geometry{verbose,letterpaper,tmargin=2cm,bmargin=2cm,lmargin=2cm,rmargin=2cm}
\usepackage{rotating}
\usepackage{color}
\usepackage{graphicx}
\usepackage{subcaption}
\usepackage{amsmath, amsthm, amssymb}
\usepackage{setspace}
\usepackage{lineno}
\usepackage{hyperref}
\usepackage{bbm}


%\usepackage{xr}
%\externaldocument{SCT-supp}

%\linenumbers
%\doublespacing
\onehalfspacing
%\usepackage[authoryear]{natbib}
\usepackage{natbib} \bibpunct{(}{)}{;}{author-year}{}{,}

%Pour les rajouts
\usepackage{color}
\definecolor{trustcolor}{rgb}{0,0,1}

\usepackage{dsfont}
\usepackage[warn]{textcomp}
\usepackage{adjustbox}
\usepackage{multirow}
\usepackage{graphicx}
\graphicspath{{../figures/}}
\DeclareMathOperator*{\argmin}{\arg\!\min}

\let\tabbeg\tabular
\let\tabend\endtabular
\renewenvironment{tabular}{\begin{adjustbox}{max width=0.9\textwidth}\tabbeg}{\tabend\end{adjustbox}}

\makeatletter

%%%%%%%%%%%%%%%%%%%%%%%%%%%%%% LyX specific LaTeX commands.
%% Bold symbol macro for standard LaTeX users
%\newcommand{\boldsymbol}[1]{\mbox{\boldmath $#1$}}

%% Because html converters don't know tabularnewline
\providecommand{\tabularnewline}{\\}

\usepackage{babel}
\makeatother


\begin{document}


\title{Principal Component Analysis of Population Genetic Data}
\author{Florian Priv\'e,$^{\text{1,2,}*}$ Keurcien Luu, Clara Albi\~nana Climent,$^{\text{2}}$ Michael G.B. Blum,$^{\text{1,}*}$ and Bjarni J. Vilhj\'almsson$^{\text{2}}$}



\date{~ }
\maketitle

\noindent$^{\text{\sf 1}}$Laboratoire TIMC-IMAG, UMR 5525, Univ.\ Grenoble Alpes, CNRS, La Tronche, France, \\
\noindent$^{\text{\sf 2}}$National Centre for Register-based Research (NCRR), Aarhus University, Denmark. \\

\noindent$^\ast$To whom correspondence should be addressed.\\

\noindent Contacts:
\begin{itemize}
\item \url{florian.prive.21@gmail.com}
\item \url{keurcien.luu@gmail.com}
\item \url{clara@econ.au.dk}
\item \url{michael.blum@univ-grenoble-alpes.fr}
\item \url{bjv@econ.au.dk}
\end{itemize}

\newpage

\abstract{

}


%%%%%%%%%%%%%%%%%%%%%%%%%%%%%%%%%%%%%%%%%%%%%%%%%%%%%%%%%%%%%%%%%%%%%%%%%%%%%%%%

\newpage

\section{Introduction}

Principal Component Analysis (PCA) has been widely used in genetics has been used for many years and in many contexts such as e.g.\ to adjust for population structure in Genome-Wide Association Studies (GWAS) by adding PCs as covariates \cite[]{price2006principal} and to detect loci under selection based on PCA \cite[]{galinsky2016fast,luu2017pcadapt}.
Much work has been devoted to developing scalable algorithms to compute PCA on data as large as the UK Biobank \cite[]{bycroft2017genome}, which is now possible thanks to software such as FastPCA (fast mode), FlashPCA2, PLINK 2.0 (approx mode), bigstatsr/bigsnpr and TeraPCA \cite[]{galinsky2016fast,abraham2017flashpca2,chang2015second,prive2017efficient,bose2019terapca}.

However, some pitfalls related to PCA of genotype data has been documented and we recall them here. This includes some possible lack of precision of computed PCs of software such as FastPCA and PLINK 2.0, which is not an issue for FlashPCA2 and bigstatsr/bigsnpr \cite[]{abraham2017flashpca2,prive2017efficient}.
This also includes capturing Linkage Disequilibrium (LD) structure in PCA instead of population structure only \cite[]{price2008long,abdellaoui2013population,prive2017efficient}. This leads to e.g.\ an over-correction of GWAS variants within long-range LD regions [REWORD/PRECISE].
Finally, another issue may arise when projecting PCs of a reference dataset to another study dataset: projected PCs are shrunk towards 0 in the new dataset \cite[]{lee2010convergence,wang2015improved,zhang2019fast}. This shrinkage makes it dangerous to use the projected PCs for analysis such as PC regression, ancestry detection and GWAS [REWORD/PRECISE].

In this paper, we derive a new implementation of PCA that can be used directly on bed/bim/fam PLINK files with some missing values, which is available in R package bigsnpr v1.0.0. Then, we recall some of the pitfalls of PCA computation on genotype data and provide efficient solutions to these issues, such as accouting for LD structure in PCA, detecting outliers [NOT INTRODUCED BEFORE] and projecting PCs to another study sample without shrinkage bias.



%%%%%%%%%%%%%%%%%%%%%%%%%%%%%%%%%%%%%%%%%%%%%%%%%%%%%%%%%%%%%%%%%%%%%%%%%%%%%%%%

\section{Material and Methods}

\subsection{PCA with missing values}

When there is no missing value, we compute the partial Singular Value Decomposition (SVD) $U \Delta V^T$ of the scaled genotype matrix $\tilde{G}_{i,j} = \frac{G_{i,j} - 2 f_j}{\sqrt{2 f_j (1 - f_j)}}$ where $G_{i,j}$ is the allele count (genotype) of individual $i$ and variant $j$, and $f_j$ is the allele frequency of variant $j$ ($2 f_j$ is the mean allele count of variant $j$). Then, $U \Delta$ are the first $K$ PC scores and $V$ are the first $K$ PC loadings, where $K$ is the number of PCs computed (e.g.\ $K=20$).

When there is some missing values, we effectively compute the partial SVD of the following matrix: $\breve{G}_{i,j} = \frac{G_{i,j} - 2 f_j}{\sqrt{2 f_j (1 - f_j) ~ p_i^{(1)} p_j^{(2)}}}$ [VERIFY THAT THIS IS CENTERED], where missing values are replaced by the mean (i.e.\ $G_{i,j} - 2 f_j = 0$ if $G_{i,j}$ is missing) and where the proportion of missing values of individual $i$, $p_i^{(1)}$, and variant $j$, $p_j^{(2)}$, are accounting for the shrinkage due to mean imputation [REWORD?].
Note that this decomposition is equivalent to the decomposition presented above when there is no missing value.

\subsection{Outlier detection in PCA}

We propose [BOF] different outlier detection techniques for PCA.

For detecting outlier variants in PCA that are due to long-range Linkage Disequilibrium (LD) regions, we use the same procedure as described by \cite{prive2017efficient}: we compute pcadapt statistics \cite[]{luu2017pcadapt} that summarises the contribution of each variant in all PC loadings. This captures both population-differentiating variants as well as long-range LD regions. We then smooth the statistics to mostly capture consecutive outliers that corresponds to long-range LD regions.

For detecting outlier samples in PCA, we use the Local Outlier Factor (LOF) on PCs \cite[]{breunig2000lof}. To make it robust to the choice of the number of nearest neighbours (kNN) and to the number of PCs used, we take the maximum value of the statistic computed with kNN $\in$ \{4, 10, 30\} and varying the number of PCs from 2 to the number of available PCs. We make use of the fast K nearest neighbours implementation of R package nabor to implement the LOF statistic efficiently \cite[]{elseberg2012comparison}.
 
For detecting samples that have a different ancestry from most of the samples in the data, we compute the pairwise orthogonalized Gnanadesikan-Kettenrin robust Mahalanobis distance of PCs \cite[]{gnanadesikan1972robust,maronna2002robust}, and do not include individuals whose log-distance is larger than some threshold determined based on visual inspection [DO IT AUTOMATIC?].

Finally, to choose the threshold of outlierness based on the previously described statistics, we use a modified version of Tukey's rule, a standard rule for detecting outliers \cite[]{tukey77}. 
The standard upper limit defined by Tukey's rule is $q_{75\%}(x) + 1.5 \cdot IQR(x)$, where $x$ is the vector of computed statistics and $IQR(x) = q_{75\%}(x) - q_{25\%}(x)$ is the interquartile range.
However, there are two pitfalls when using Tukey's rule. First, Tukey's rule assumes a normally distributed sample. For example, when the data is skewed, it does not work that well. We account for skewness using the medcouple \cite[]{hubert2008adjusted}. Standard Tukey's rule also uses a fixed coefficient (1.5) that does not account for multiple testing, which means that for large samples, you will almost always get some outliers if using 1.5. 
To solve these two issues, we implement \texttt{tukey\_mc\_up} in R package bigutilsr and use it here, which accounts both for skewness and multiple testing by default.

\subsection{Projecting PCs from a reference dataset}

We provide a subsetted version of the 1000 genomes (1000G) project data \cite[]{10002015global,meyer2019genotype}.
Variants are restricted to the ones in common with HapMap3 and UK Biobank \cite[]{international2010integrating,bycroft2017genome}. 
Moreover, we apply some quality control filters; we remove variants having a minor allele frequency < 0.01, variants or individuals with more than 10\% missing value, variants with P-value of the Hardy-Weinberg exact test < $10^{-50}$, and non-autosomal variants. 
To remove related individuals with first-degree relationship or more, we apply KING cutoff of 0.177 to the data \cite[]{manichaikul2010robust,chang2015second}.
This results in 2493 individuals of the 1000G project (phase 3) in PLINK bed/bim/fam format. 
Resulting PLINK files and R code to generate these files are available at \url{https://doi.org/10.6084/m9.figshare.9208979.v2}. 
To easily download these data, we provide a function called \texttt{download\_1000G} in R package bigsnpr.

There are 4 steps to project PCs of a reference dataset (e.g.\ 1000G) to a target genotype data:
\begin{enumerate}
\item matching the variants of each dataset, including removing ambiguous alleles [A/T] and [C/G], and matching strand and direction of the alleles;
\item computing PCA of the reference dataset using the matched variants only;
\item projecting computed PCs to the target data;
\item adjusting projected PCs for shrinkage.
\end{enumerate}
We implement these steps in function \texttt{bed\_projectPCA} of package bigsnpr.

For step 4, we implement a shrinkage-bias adjustment method that combines the two methods proposed by \cite{zhang2019fast}. First, for only a subset of target individuals, we compute the Online Augmentation, Decomposition, and Procrustes (OADP) transformation (using $K''=K'=K$). For these individuals, we regress the new PC scores resulting from OADP by the projected PC scores without adjustement; this gives us shrinkage-adjustment factors that we use to correct projected PC scores for all target individuals.

%This requires computing all eigenvalues of the reference dataset, which is relatively fast to compute for the 2493 individuals of the 1000G. However, it can be computionally expensive if the reference data used includes many more individuals [BETWEEN O($N^2$) AND O($N^3$); MENTION IT?].

%%%%%%%%%%%%%%%%%%%%%%%%%%%%%%%%%%%%%%%%%%%%%%%%%%%%%%%%%%%%%%%%%%%%%%%%%%%%%%%%

\section{Results}



%%%%%%%%%%%%%%%%%%%%%%%%%%%%%%%%%%%%%%%%%%%%%%%%%%%%%%%%%%%%%%%%%%%%%%%%%%%%%%%%

\section{Discussion}

[LIMITATION: RUN MULTIPLE TIMES / FAIL TO SPEED UP]



\subsection{Conclusion}




%%%%%%%%%%%%%%%%%%%%%%%%%%%%%%%%%%%%%%%%%%%%%%%%%%%%%%%%%%%%%%%%%%%%%%%%%%%%%%%%

%\newpage

\section*{Funding}

\section*{Acknowledgements}

This research has been conducted using the UK Biobank Resource under Application Number 25589.

~\\ \textit{Conflict of Interest:} none declared.

%%%%%%%%%%%%%%%%%%%%%%%%%%%%%%%%%%%%%%%%%%%%%%%%%%%%%%%%%%%%%%%%%%%%%%%%%%%%%%%%

%\newpage

\bibliographystyle{natbib}
\bibliography{refs}

%%%%%%%%%%%%%%%%%%%%%%%%%%%%%%%%%%%%%%%%%%%%%%%%%%%%%%%%%%%%%%%%%%%%%%%%%%%%%%%%

\newpage
\section*{Supplementary Materials}



%%%%%%%%%%%%%%%%%%%%%%%%%%%%%%%%%%%%%%%%%%%%%%%%%%%%%%%%%%%%%%%%%%%%%%%%%%%%%%%%

\clearpage

\vspace*{1em}

\begin{figure}[!htpb]
%\centerline{\includegraphics[width=0.8\textwidth]{AUC-simu2.pdf}}
\caption{}
\label{fig:AUC-simu2}
\end{figure}


\vspace*{1em}

% latex table generated in R 3.5.2 by xtable 1.8-4 package
% Wed Jun 19 15:19:38 2019
\begin{table}[!htpb]
\caption{AUC values on the test set for simulations with less well imputed variants (mean [95\% CI] from $10^4$ bootstrap samples). These results are plotted in figure \ref{fig:AUC-simu2}.\label{tab:AUC-simu2}}
\vspace*{0.5em}
\centering
\begin{tabular}{|l|c|c|c|c|}
\hline
Scenario & stdCT & maxCT & SCT & lassosum \\ 
\hline
100  & 77.4 [76.0-78.8] & 83.9 [83.4-84.4] & 83.1 [82.6-83.6] & 80.1 [79.5-80.8] \\ 
10K  & 69.4 [68.4-70.5] & 73.0 [72.5-73.4] & 72.9 [72.5-73.3] & 71.2 [70.6-71.7] \\ 
1M   & 64.0 [63.6-64.4] & 64.0 [63.6-64.4] & 62.7 [62.3-63.0] & 64.1 [63.3-64.8] \\ 
2chr & 70.0 [68.8-71.2] & 74.4 [73.6-75.2] & 78.5 [77.9-79.1] & 73.2 [72.5-73.8] \\ 
err  & 67.0 [66.0-68.1] & 68.6 [67.7-69.5] & 69.5 [68.9-70.1] & 65.6 [64.9-66.3] \\ 
HLA  & 74.8 [72.9-76.3] & 75.3 [73.5-76.9] & 76.4 [74.5-78.0] & 75.8 [74.2-77.2] \\ 
\hline
\end{tabular}
\end{table}

%%%%%%%%%%%%%%%%%%%%%%%%%%%%%%%%%%%%%%%%%%%%%%%%%%%%%%%%%%%%%%%%%%%%%%%%%%%%%%%%


\end{document}
